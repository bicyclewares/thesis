%!TEX root = /Users/will/Documents/Academic/MA 470 -THESIS/THESIS/thesis.tex
\chapter*{Preface}
\section*{A Word About The Format of This Thesis}
One of the pedagogical aims of this thesis is to be clear and reader-assistive during its presentation.  To that end, I have taken several liberties in the formatting of this thesis.  Figures, equations, definitions, theorems and examples all share one numbering scheme in hopes that it will make them easier to locate. Where important \defmarginnoindex{definitions} appear, they are clad in italics and sit in a box on the margin to make them easier to find.  When\refmarginnoindex{definitions} definitions or equations are referred to later in the text, an assistive link will appear in the margin to avoid index-fingering.  In addition, the wired reader will find a searchable pdf version of the text at the following url:
\begin{center}
	\emph{http://wcrawford.org/thesis}
\end{center}
All references in this pdf are hyper-textual (clickable).

\section*{Acknowledgments}
Thanks first and foremost go to my advisor, Jim Fix, for his insights, guidance and willingness to explore the many systems and ideas that led the creation of this thesis.  None of this could have happened without his help.  I also thank my professors and colleagues for their willingness to read versions of this thesis and provide feedback for my ideas.  Finally, a huge portion of this thesis relies directly on the work of Robin Milner and many others who have expanded on what he set in motion.  To the extent that anything novel is presented in this thesis, it relies heavily on the great minds that have inspired and challenged my thinking of distributed computing.

I would also like to thank my friends and family for their continuing support, compassion and understanding.  In particular I thank my parents, with whom neither my thesis nor the education proceeding it would have been possible.