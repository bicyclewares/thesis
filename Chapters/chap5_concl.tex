\chapter{Conclusions}
In Chapter \ref{sync_and_dist_sys} we said with Palamidessi that no uniform encoding of the synchronous \picalc\ into the asynchronous \picalc preserving reasonable semantics exists.  This is a very strong result.  Weakening either of the requirements that Palamidessi assumes seems like it might produce an encoding not rigorous enough to study.  
For this reason, the fully expressive synchronous \picalc\ seems like a better candidate for formal study than the asynchronous \picalc.

However, we need still consider the implementation of a synchronous \picalc.  
On distributed systems, we have only asynchronous sending available to us.  
Hence, it also seems useful to study the asynchronous \picalc\ since it models these systems more accurately than a synchronous model.
For the study of distributed systems, rather than showing the asynchronous to be not worth our time, Palamidessi's separation result raises the question of whether we should be considering \emph{synchronous} calculi. 

Ideally, we'd like the best of both worlds.  The expressiveness of the synchronous \picalc\ allows us to solve a large class of problems much more easily and clearly.
We saw just how useful the synchronous \picalc\ can be for expressing distributed systems in our extended mobile phone network example in the Chapter \ref{Introduction}.
We could have modeled this system in the asynchronous \picalc, but it would have involved a convoluted mess of acknowledgement channels just to express the necessary ordering of events in the system.
Hence, the last chapter looked at some of the more implementation-minded encodings of the synchronous \picalc\ in the asynchronous \picalc, and to what extent we need to relax Palamidessi's requirements to allow these encodings.

The creators of Pict, the Join-calculus, and other implementations based on the \picalc\ all decided to have their primitives support only asynchronous communication, while synchronous communication is made available overtop of this via a library or higher-level language.  
This these greatly simplifies implementation, resulting in a cleaner, more efficient core language.  
The summation operator in particular is difficult and expensive to fully simulate.  
In the implementation of Pict, for example, David Turner notes \cite{turner96} that ``the additional costs imposed by summation are unacceptable.''.  
Turner goes on to say that \emph{essential} uses of summation are infrequent in practice.

Speaking in an interview on developing the \picalc, Robin Milner notes \cite{miln03}:
\begin{quote}
That was to me the challenge: picking communication primitives which could be understood at a reasonably high level as well as in the way these systems are implemented at a very low level...There's a subtle change from the Turing-like question of what are the fundamental, smallest sets of primitives that you can find to understand computation...as we move towards mobility... we are in a terrific tension between (a) finding a small set of primitives and (b) modeling the real world accurately.
\end{quote}
This tension is quite evident in the efforts of process algebraists to find the `right' calculus for modeling distributed systems.  
While the synchronous \picalc\ is more elegant and fundamental, actual implementations must commit to asynchronous communication as their primitives.  
Hence, which we choose as a model depends in part on our goals.  
In any case, it is evident that by limiting ourselves to smaller calculi, many useful new concepts and structures arise in order to solve the problems posed by asynchronous communication.  
While these structures might not belong in the `smallest set of primitives', they are useful for bringing the power of the \picalc\ to a model that more closely resembles the implementation of distributed systems.