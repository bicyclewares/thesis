
	%pi-calc commands
			
			
	\newcommand{\picalc}{%
		\texorpdfstring{$\pi$}{pi}-calculus}
	% \newcommand{\HUGE}{\fontsize{43}{54}\selectfont}
	% \newcommand{\PICALC}{%
	% 	\texorpdfstring{{\Huge{$\pi$}}}{pi}-Calculus\ %
	% }
	\newcommand{\exports}[1]{%
	\ensuremath{({#1})}%
	}
	\newcommand{\Picalc}{%
		\texorpdfstring{$\pi$}{pi}-Calculus}
	
	\newcommand{\send}[2]{%
	 	\ensuremath{{#1}!\langle{#2}\rangle}%
	}
	\newcommand{\sends}[2]{%
	 	\ensuremath{{#1}!{#2}}%
	}
	\newcommand{\ssend}[2]{%
	 	\ensuremath{{#1}!\langle{#2}\rangle.}%
	}
	\newcommand{\receives}[2]{%
	 	\ensuremath{{#1}?{#2}}%
	}
	\newcommand{\receive}[2]{%
	 	\ensuremath{{#1}?({#2}).}%
	}
	\newcommand{\receivenodot}[2]{%
	 	\ensuremath{{#1}?({#2})}%
	}
	\newcommand{\new}[1]{%
	 	\ensuremath{\text{new}({#1}).}%
	}
	\newcommand{\pif}[1]{%
		\ensuremath{\text{if } {#1}}%
	}
	\newcommand{\pthen}{%
		\ensuremath{\text{ then }}%
	}
	\newcommand{\pelse}{%
		 \ensuremath{\text{ else }}%
	}
	\newcommand{\ptrue}{%
		\ensuremath{\text{\emph{true}}}%
	}
	\newcommand{\pfalse}{%
		\ensuremath{\text{\emph{false}}}%
	}
	\newcommand{\pretry}{%
		\ensuremath{\text{\emph{retry}}}%
	}
	\newcommand{\pstop}{%
	\ensuremath{\mbox{\emph{stop}}}%
	}
	\newcommand{\comp}{%
		\ensuremath{\ |\ }%
	}
	\newcommand{\tuple}[1]{%
	\ensuremath{\overline{#1}}%
	}
	\newcommand{\rec}[1]{%
		\ensuremath{\text{rec }{#1}.}%
	}
	\newcommand{\defequals}{%
	 \ensuremath{\overset{\mbox{\tiny{def}}}{=}}%
	}
	\newcommand{\encode}[1]{%
	 \ensuremath{\llbracket{#1}\rrbracket}%
	}
	\newcommand{\subst}[2]{%
	 \ensuremath{\llbracket \nicefrac{#1}{#2}\rrbracket}%
	}
	\newcommand{\pdef}{\ensuremath{\Leftarrow}\ }
	\newcommand{\sys}[1]{\ensuremath{\mathbb{#1}}}
	\newcommand{\sequiv}{\ensuremath{\equiv}}
	\newcommand{\pred}{\ensuremath{\longrightarrow}}
	\newcommand{\evolves}[1]{\ensuremath{\overset{\underrightarrow{\ {#1}\ }}{}}}
	
	\newcommand{\preds}{\ensuremath{\cdotp\!\cdotp\!\cdotp\!\longrightarrow}}
	%end pi-calc commands